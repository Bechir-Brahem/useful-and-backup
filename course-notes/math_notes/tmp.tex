%&latex
\documentclass[10pt]{article}
\usepackage[utf8]{inputenc}
\usepackage{amssymb}
\usepackage{amsmath}
\usepackage{geometry}
\renewcommand{\baselinestretch}{1.2}
\geometry{
	a4paper,
	total={170mm,257mm},
	left=30mm,
	top=30mm,
}
\begin{document}
	
	%+Title
	\title{discrete math}
	\author{Bechir Brahem}
	\date{\vspace{-5ex}}
	\maketitle
	
	
	\par\noindent\rule{\textwidth}{3pt}
	\section{Points About Logic\ }
		\subsection{Proposition}
			\textbf{Definition. }A proposition is a statement (communication) that is either true or
			false. 
		\subsection{Predicate}
			\textbf{Definition.} A predicate can be understood as a proposition whose truth depends on the value
			of one or more variables. 
		\pagebreak
		
	\par\noindent\rule{\textwidth}{3pt}
	\section{Induction,WOP and Invariants\ }
		\subsection{Well Ordering Principle}
			\underline{\textbf{Theorem.}} Every nonempty set of nonnegative integers has a smallest element.
			\paragraph{scheme of the proof. }
				Principle, you can take the following steps:\\
				More generally, to prove that “P(n) is true $\forall n \in \mathbb{N}$ .” using the Well Ordering
			\begin{itemize}
			
				\item Define the set, C , of counterexamples to P being true. Namely, define \begin{center}$C ::= \{n \in \mathbb{N} | P(n) is false \}$\end{center}
				\item Use a proof by contradiction and assume that C is nonempty.
				\item By the Well Ordering Principle, there will be a smallest element, n, in C .
				\item Reach a contradiction (somehow)—often by showing how to use n to find another member of C that is smaller than n. (This is the open-ended part of the proof task.)
				\item Conclude that C must be empty, that is, no counterexamples exist. QED$\blacksquare$
			\end{itemize}
			\textbf{examples.} it can be used to prove the sum of integers is:$\sum_{0}^{n} k =\frac{n(n+1)}{2}$
			 or to show that every integer is a product of primes.(\emph{Fundamental theorem of arithmetic without uniqueness})
			 \paragraph{proof.} 
				 let the predicate P(n):=" $\forall n \in \mathbb{N} | n=p_{1}p_{2} ... p_{k}$ "
				let S={n | P(n) is false}
				n is a positive integer, we assume that S$\neq\emptyset$ by the WOP S has a smallest element $n_{0}$.
				 if $n_{0}$ was a prime it would be in S so $n_{0}$ is not a prime $\Rightarrow$ $n_{0}$=ab
				 $\Rightarrow 0 < a,b < n_{0}$ so a=$p_{1}p_{2} ... p_{k}$ and b$=q_{1}q_{2} ... q_{k}$
				  where $p_{i}$ and $q_{i}$ are primes
				 $ \Rightarrow $ ab$ \in S $$ \Rightarrow $ $ n_{0}\in S $ absurd.
				 hence we have P(n)
		 \subsection{Induction}
			 \paragraph{scheme of the proof.}
			 let P(n) be the predicate we want to provein S.\\If:\\
			
			
			\[
			\begin{cases}
			P(n_0)\ is\ true\\
			P(n)\Rightarrow\ P(n+1) 
			\end{cases}
				\]
			Then:\\
			P(m) is true $ \forall m \in S $
		\subsection{Strong Induction}
			\paragraph{scheme of the proof.}
			let P(n) be the predicate we want to prove in S.\\If:\\
			\[
			\begin{cases}
			P(n_0)\ is\ true\\
			\forall n \in S,\ we\ have\ P(0),P(1)...P(n)\Rightarrow P(n+1)
			\end{cases}
			\]
			Then:\\
			P(m) is true $\forall m\in$ S
			\paragraph{examples.}P(n)::="\emph{Every integer greater than 1 is a product of primes.}"\\
				2 is a prime so we have P(2).\\
				assuming we have P(0)...P(n),\\
				if n+1 is a prime then P(n+1)\\
				if n+1 is composite then n+1=ab such that $1<a,b<n$. so we have P(a) and P(b) and so n+1 is a prime.
				$ \Rightarrow P(n+1)$\\
				so $ \forall m \in S,P(m) $
		\subsection{Invariants}{
			\paragraph{}{
				The idea of the proof by invariant is that for some process there is a proprety X that remains constant for every state.}
			\paragraph{example.}{
				say that a robot on a grid can only move diagonally.from the initial position (0,0) the robot can go to (1,1),(-1,1),(1,-1),(-1,-1)\\
				\emph{claim. a robot can never reach (1,0) if (0,0) is its initial position.}\\
				\textbf{\underline{proof.}} the invariant:\emph{if (0,0) is the initial state then whatever position the robot gets into (x,y) x+y is even.}\\
				base case: 0+0 is even.\\
				induction:if the robot is in position (x,y) we assume that x+y is even then the next position (a,b) will be:(x+1,y+1) or (x-1,y-1) or (x-1,y+1) or (x+1,y-1). and so in every case a+b is even.\\
				for any position (x,y) to be reachable x+y must be even.\\and so (1,0) is not reachable from (0,0)
				
				
			}
			\paragraph{scheme of the proof}{
			In summary, if you would like to prove that some property X holds for every
			step of a process, then it is often helpful to use the following method:
			\begin{itemize}
				\item Define P(t) to be the predicate that X holds immediately after step t .
				\item Show that P(0) is true, namely that X holds for the start state.
				\item show that:
				\[
				\forall t\in\mathbb{N},P(t)\Rightarrow P(t+1)
				\]
			\end{itemize}
			}
		}
	\pagebreak
	\par\noindent\rule{\textwidth}{3pt}
	\section{Number Theory}{
		\subsection{Math theory}{
			\subsubsection{Basics}{
				\paragraph{Definition. }
					a divides b (notation a | b) iff there is an integer k such that
					\begin{center}ak = b\end{center}
				}
				\paragraph{Theorem. }Let n and d be integers such that d $ \neq $ 0.
				Then there exists a unique pair of integers q and r, such that:
				\begin{center}\large n=dq+r , $ 0\leq r \leq |d|$\end{center}
				\paragraph{Euclid Algorithm. }{for b$ \neq $0 
					\begin{center}
					\large	gcd(a,b)=gcd(b,rem(a,b))
					\end{center}
				}
				\paragraph{Bezout Theorem. }{The greatest common divisor of a and b is a linear combination
					of a and b. That is:
					\begin{center}
						\large gcd(a,b)=sa + tb
					\end{center}
					for some t,s}
			}
			\subsubsection{Prime Numbers. }{
				\begin{itemize}
					\item\textbf{Twin Prime Conjecture} There are infinitely many primes p such that p + 2 is also a prime
					\item\textbf{Conjectured Inefficiency of Factoring} Given the product of two large primes n=pq, there is no efficient procedure to recover the primes p and q. That is,no polynomial time procedure. Best solution so far 
					\begin{center}\large$ e^{1.9(ln\ n)^\frac{1}{3}(ln\ ln\ n)^\frac{2}{3}} $\end{center}
					\item\textbf{Goldbach’s Conjecture} every even integer greater than two is equal to the sum of two primes.
				\end{itemize}
				\paragraph{Prime Distribution. }
				{
					\begin{center}$\pi$(n)::=card(\{p, p is prime and 2$\leq$p$ \leq $n     \})\end{center}
				}
				\paragraph{Prime Number Theorem. }{
					$$\lim_{x\to\infty}\frac{\pi(n)}{n/ln(n)}=1$$
				}
				\paragraph{Fundamental Theorem of Arithmetic. }{Every positive integer is a
				product of a unique weakly decreasing sequence of primes.}
			}
			\subsubsection{Modular Arithmetic}{
				On the first page of his masterpiece on number theory, Disquisitiones Arithmeticae,
				Gauss introduced the notion of “congruence.” Now, Gauss is another guy who
				managed to cough up a half-decent idea every now and then, so let’s take a look
				at this one. Gauss said that a is congruent to b modulo n iff n | (a-b). This is
				written
				$$ a\equiv b (mod\ n) $$
				\begin{equation}\label{key}
				 a\equiv b (mod\ n)\ \ \Leftrightarrow\ \ rem(a,n)= rem(b,n) 
				\end{equation}
				\\We have $a\equiv b (mod\ n)$ and $c\equiv d (mod\ n)$ then:
				$$ a+c \equiv b+d (mod\ n)$$ $$ a b\equiv cd (mod\ n)$$ 
			\subsubsection{The Ring $ \mathbb{Z}_n $}{
				$ \mathbb{Z}_n $=\{r $|$ for a$ \in \mathbb{Z}, a\equiv r (mod\ n) $\}	 for example $\mathbb{Z}_n $=\{0,1,2...n\},\\ 
				we define $ r=a\pmb{+_n} b $ :$(a,b)\in\mathbb{Z}_n $ $\rightarrow\ $$r\in\mathbb{Z}_n $ such that a+b$ \equiv $ r (mod n)\\
				for example 5 $ +_7 $ 4 = 2 \\
				we define $ r=a\pmb{\cdot_n} b $ :$(a,b)\in\mathbb{Z}_n $ $\rightarrow\ $$r\in\mathbb{Z}_n $ such that a$ \cdot $b$ \equiv $ r (mod n)\\
				for example 5 $ \cdot_7 $ 4 = 6
				
				
				}
			}
			
		
	
		
	
	}


\end{document}


